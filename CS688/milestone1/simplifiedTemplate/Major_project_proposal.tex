\documentclass[12pt]{extarticle}
\usepackage[utf8]{inputenc}
\usepackage{cite}

\title{Human Pose Estimation: PoseNet}
\author{Flavio Amurrio-Moya, G00593001}
\date{famurrio@gmu.edu}

\begin{document}

\maketitle

\newpage

% Deliverables:
%     (1) code with a readme file that explains how to use the code;
%     (2) a report explaining the specific task, experiments, and results.

%%%%%%%%%%%%%%%%%%%%%%%%%%%%%%%%%%%%%%%%%%%%%%%%%%%%%%%%%%%%%%%%%%%%%%%%
\section{Source Paper}
Section: "Source paper": full reference of the paper (authors, title, venue and year of publication)

%%%%%%%%%%%%%%%%%%%%%%%%%%%%%%%%%%%%%%%%%%%%%%%%%%%%%%%%%%%%%%%%%%%%%%%%
\section{Source Code}
Section: "Source code": language used; list of libraries used and for which task; description of the code you wrote and for each tasks

%%%%%%%%%%%%%%%%%%%%%%%%%%%%%%%%%%%%%%%%%%%%%%%%%%%%%%%%%%%%%%%%%%%%%%%%
\section{Methodology}
Section: "Methodology": Brief and clear description of the algorithm you have implemented

%%%%%%%%%%%%%%%%%%%%%%%%%%%%%%%%%%%%%%%%%%%%%%%%%%%%%%%%%%%%%%%%%%%%%%%%
\section{Experiments}
% Section: "Experiments":

\subsection{Experiments Description}
- Description of the experiments you have conduced

\subsection{Source Paper Results}
- Clearly identify the results in the source paper that you have replicated or tried to replicate in case you had to make adjustments

\subsection{Dataset Used}
            - Description of the data used and link to the data: number of samples, number of features, number of classes, etc.                   - Provide a link to the data.

\subsection{Evaluation Measure}
             - Evaluation measures used

\subsection{Training and Testing}
             - Discuss how training and testing were conducted; cross-validation techniques; numbers of runs, etc.

\subsection{Experiment Results}
             - Tables and/or Plots of the results

\subsection{Analysis}
             - Presentation, Discussion, and Analysis of the results



% \section{Proposal}
% I will be planning on using Humam Pose Estimation to gather data on different type
% of human activities. Using this data, I would then build a Machine Learning
% Model and/or Deep Neural Network with the human pose data which would allow me
% to classify body positions. One possible application is to have a model that
% would detect when a student has their hand raised. This would turn on a light on the instructor's desk. This could be done for both in person and
% online courses. Another possible application is to detect human positions when
% playing games such as charades and have those positions be described to a visually
% impaired person.

% \section{Survey}

% \subsection{Introduction}
% In recent years, the application of Human Pose Estimation has progressed
% exponentially in terms of speed and accuracy. From being able to detect simple
% 2D poses of one person at the time to real-time multi-person 3D Human Pose
% detection in videos, this rapid evolution is mainly thanks to the advances of
% Deep Neural Networks, which help with the detection of body part sections and
% joints within the human body. Human Pose Estimation originally began as a means
% of basic object detection before its broader application into healthcare,
% entertainment, and videography in general. This document will seek to explain
% Human Pose Estimation's history, it's conception and progression according to
% the latest research, as well as its shortcomings in current applications.

% The origin of Pose Estimation was first hinted at by Martin A. Fischler and
% Robert A Elsachelager \cite{FischlerM.A1973TRaM} in 1973 in the form of matching
% Pictorial Structures, where, given a description of an object, one could find
% the object in a picture. One of the applications mentioned on
% \cite{FischlerM.A1973TRaM} was to perform some variance of facial detection.
% During this time, there was a strong need for object detection which prompted
% hastened research in Pictoral Structures. As technology advanced, more intricate
% and complex renditions of the human anatomy were required. This would later
% evolve beyond the scope of mere physical appearance to human behavior and movement as
% well, which will be further explained as we delve into the more technical aspect in the below
% survey.

% \subsection{Survey}
% To achieve a realistic depiction of the human form, Human Pose Estimation
% requires, in broad terms, two main steps - preprocessing and body parts parsing
% \cite{LiuZhao2015Asoh}. It also requires training deep learning models to
% predict the poses. We will further elaborate the method preprocessing in detail
% below.

% Before the popular models and approaches to solving this problem can be explained, we
% must discuss that pose estimation falls under two different methodologies,
% starting with the top-down approach. In the top-down approach, human detection
% is done using CNNs in two steps. First, much like the object detection
% algorithms, the framework will first detect bounding boxes around the humans.
% Later it involves estimating the pose inside of the boxes. Estimating poses
% requires finding important coordinates on a human body. These would include key
% points, such as knees, elbows, etc. For example, in the much widely used
% datasets, such as the COCO Dataset \cite{guler2018densepose}, 17 coordinates are
% used. In other research papers, and datasets, different numbers of keypoints
% could be used. These coordinates are paired with an extra field, which indicates
% whether the coordinate is visible or not. There are often occlusions to the body
% parts and it is the neural network’s job to infer the points regardless, and
% thus the training data must contain the points.

% In the bottom-up, CNNs are also used to solve the detection problem. However,
% the approach will first estimate all the key points in an image and then go on
% to classify which key points are for a person \cite{CaoZhe2021ORM2}. This helps
% in both “performance and efficiency”because top-down will estimate poses per
% person and is a two step process. There are many open source models that resolve
% this problem, and we will discuss the most recognized and state of the art model
% OpenPose \cite{CaoZhe2021ORM2}. The methods described here fall under the
% bottom-up, as do most of the state of the art approaches such as Regional
% Multi-Person Pose Estimation (AlphaPose) \cite{DBLP:journals/corr/FangXL16} .
% These models have been optimised to that they could be used in browser and/or
% mobile without the use of a GPU.

% \subsection{Prepocessing}
% Preprocessing can be generously defined as the work that must be done in order
% to best optiomized the data that will be used for the Machine Learning Model. In
% this discipline in particular, it involves the initial division of the human
% body into several parts \cite{papandreou2018personlab}, taken with the exact
% measurements as expressed by a camera's calibration. The camera, in this
% instance, requires varying levels of perspective for it's beginning depiction,
% before it can be compiled through a process known as Data calibration
% \cite{StollC2011Famt}. A popular body capturing device used to gather data is
% Microsoft's Kinect - most likely due to its dominating presence in the market
% and low cost \cite{Diego-MasJoseAntonio2014UKsi}.

% Another proponent of preprocessing involves body localization
% \cite{LiuZhao2015Asoh}. Body localization, as the name suggests, involves using
% a human detection model to recognize and locate a humanoid figure. Perhaps a
% robust example of this could be asking an Machine learning model which could
% distinguish a human body while it stands in front of a painted mosaic. This
% method then lends itself into the usage of Human detection, which involves a
% similar concept in the recognition of a person's location and space in a given
% image. Much like other vision algorithms such as object classification,
% segmentation, and object detection, deep learning has improved upon traditional
% methods for pose estimation.

% Human detection is done using a CNN in two steps. First, much like the object
% detection algorithms, the framework will first detect bounding boxes around the
% humans. Later it involes estimating the pose inside of the boxes. Estimating
% poses require finding important coordinates on a human body. These would include
% key joints, such as knees, elbows, and wrists.

% The final method of preprocessing involves Background Subtraction. Background
% Subtraction involves removal of data not needed. This can be achived by the
% usage of dense regions \cite{BMVC.24.34} where lighting and context change can
% be utilized to distinguish between what is in the foreground and what is in the
% background.

% Note that the above processes are components of preprocessing, but are not
% inherent steps required in any particular order (at least according to the
% resources gathered). It is an educated estimation that while each method
% contains its strengths and weaknesses, a combination of the methods summarized
% above would lead to the greatest dataset needed to create and image. When
% attempting to replicate the human form, it is important to note that each
% process requires the same need for feature extraction \cite{LiuZhao2015Asoh}.

% The second, and most admittedly most recognized portion of HPE is body parts
% parsing \cite{LiuZhao2015Asoh}. This is the process of identifying each section
% of a person body. There are four categories for this process which include 2D
% and 3D single and multi-person parsing. Each one of these categories have
% specific methods for feature extraction, appereance model and structure models.
% The compilation of these models and tooling all work to create a holistic image.

% Out of the four categories of body parts parsing, two approaches by Chen/Yuille
% \cite{ChenXianjie2014APEb} and Thompson \cite{TompsonJonathan2014JToa} are
% perhaps the most popular, efficient and accurate. This can be surmised by the
% results of the subsequent benchmark performed by \cite{LiuZhao2015Asoh}. Results
% were strikingly similar and equally positive.

% \subsection{Conclusion and Final Remarks}

% Human Pose Estimation, like with all disciplines in technology, is a constantly
% evolving process - and along with each of it's accomplishments, lies the
% existence of controversy and failure. With respect to AI (and it’s inferred
% infallibility when given the right dataset), we then must defer to controversies
% that are imposed mostly by human error, bias, and methodology. That is to say,
% that most of the controversy surrounding Human Pose Estimation is caused by the
% application of it’s use, rather than by the technology itself. One striking
% topic that has arose since the emergence of HPE is fitness and the tools we use
% to improve it. The question then arises as to how we should cater to a more diverse
% group of users. As Maksym Tatriants \cite{tatariants_2020} cite’s, “fitness is
% on trend,”and HPE has entered the playground as a helpful tool in analyzing a
% user’s posture during regular exercise. While the idea of this application is
% beneficial, it has been found to have several shortcomings in real world
% application.

% Take Maksym Tatariants, author of "Challenges of Human Pose Estimation in
% AI-Powered Fitness Apps" \cite{tatariants_2020}, and his experience practicing a
% simple squat using a fitness AI model as an example. In his experience,
% Tatariants remarks that before the exercise session can even begin, there are
% already shortcomings in its attempt to guide the user towards the correct
% posture. “Men and women's bodies are physiologically different,” he explains.
% “If the model was trained only on men's images, it will return accurate results
% for only male users but not for females.” This also brings into question the
% height and weight of dataset the AI was given versus the same measurements of
% the user performing the exercise. In this day and age, what is considered
% average height and weight? How broad of a range in this matter can be determined
% to be safe before the exercise can be performed correctly? These are popular
% media questions that, while they do may neccessarily contribute to the
% advancement of Human Pose Estimation, may very well effect it’s application and
% use in the future.

% In spite of this, Human Pose Estimation and it’s achievements in all fields, not
% just fitness, will continue to push it forward. As the public becomes gradually
% more comfortable with the assistance of AI-rendered HPE in their everyday lives,
% the more unique and diverse datasets will become to provide a human assisted
% application that will benefit everyone.


\bibliographystyle{plain}
\bibliography{M335}

\end{document}
